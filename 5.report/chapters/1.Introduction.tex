%\lipsum[4-4]
This chapter presents the context, motivation and document structure of the activity report of the work "monitoring photovoltaic power converters using Wireless Sensor Networks". 

\section{Context and motivation of PhD}

The railway system is responsible for 1.3\% of entire European energy consumption, \cite{iea-uic2016}. 
The discussion of the energy efficiency in railways is a grown topic due to its contribution to the global energy consumption.

The energy efficiency analysis and management requires a detailed mapping of the energy consumption/generation in the railway system. 

This detailed mapping of the energy flows should include, not only the rolling stock level but also the traction substations and the auxiliary services.

The knowledge of all the load curves permits the load prevision, peak shaving and energy cost optimization for all global railway system.

\section{Context and motivation of monitoring PV converters using WSN's}


This activity report is inserted in the scope of a microgeneration project of the SYSTEC Research Unit.
%Figure 1 in attachment presents the main architecture of this project. 
Currently, several work has been done in this project to implement a monitoring subsystem. This work focuses on data collection from each power converter and also on its control by sending references and actions. At the moment of this proposal, no work has been done to implement the monitoring feature in the PV converters. This way, a wireless network implementation is proposed to monitor the PV power converters.

The PV converter uses a non-isolated high gain DC/DC topology and has a PIC32 microcontroller which implements a MPPT algorithm in the control loop. It is currently possible to connect a device with wireless capabilities to the PV power converter. 

The converter is able to provide power and data. The data collected from the monitoring device is sent to an aggregator node for post processing and data storage. Currently, several work has already been done to have a reliable wired data aggregator/concentrator.
% (figure 2).
This subsystem is also responsible for sending user commands to the power converters and presenting the user interface in a webpage remotely accessed via a web browser. 




\section{Document structure}

This document is divided in 5 chapters, each of them incorporate the relevant subsections to present the subjects mentioned. 
%contains several subsections according to the subjects mentioned.

\begin{table}[!h]
    \label{tb:struct}
    \centering
    \caption{Document structure}
    \vspace{0.2em}
    \begin{tabular}{c|l}%{C{2cm}|C{9cm}}
    \textbf{Chapter} & \textbf{Title}                    \\ \hline
    1       &                   Introduction             \\ \hline
  %  2       &                   Railways Remote Monitoring Systems       \\ \hline
    2       &                   System Specification    \\ \hline
    3       &                   Development of Wireless Monitoring System    \\ \hline
    4       &                   Results and Discussion    \\ \hline
    5       &                   Conclusions and Future Work      \\ 
    \end{tabular}
\end{table}