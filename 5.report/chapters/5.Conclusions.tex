%\lipsum[1]

With this work, we propose and implement a solution for the wireless monitoring of power converters. The main challenge of this wireless monitoring is to obtain energy data of a system in a harsh environment. The motivation for this work is to study the implementation of a wireless monitoring system in railway environment, based on the outcomes of the implementation of such monitoring system in a renewable energy generation system.

The system specification was defined, with the construction of a System Requirement Specification (SRS) document as the main outcome. In addition, it was performed an overview on existing wireless communication systems, a market survey on available technologies, and the protocols, standards and communication KPI's was raised.

A specific technology was selected, on the family of wireless MCU's. The implementation stage complies with the definition of the hardware and software architecture. The communication results validates the acquisition of electric measurements from multiple PV power converters. 

The lack of data results is clear. Any of the communication KPI's was not evaluated due to the need of further development. This further development depends on a new solution for the interface between the power converter and the wireless node, since the communication link is highly affected by the noise of power converter.

For future work is clear the need of evaluating all the proposed communication KPI's. This task requires a new electronic board to interface the remote node. Complementary, the data concentrator should be improved to implement a local database and a serial request-response protocol to exchange data with the microgeneration system master.

We conclude that the methodology followed and the proposed solution validates the objective of this work, which is the wireless monitoring of a power converter.