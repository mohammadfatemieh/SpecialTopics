\documentclass{report}%{IEEEtran}
\usepackage[utf8]{inputenc}
\usepackage[square]{natbib}
\usepackage{graphicx}
\usepackage{framed}
\usepackage{multirow}
\usepackage{lipsum}  
\usepackage{verbatim}

\usepackage[oneside,width=17.5cm,height=24cm,left=2cm]{geometry}
%\usepackage[nolist,nohyperlinks]{acronym}

\usepackage[]{nomencl}

%\usepackage{biblatex}
%\addbibresource{references.bib}

\usepackage{array}
\newcolumntype{C}[1]{>{\raggedright\let\newline\\\arraybackslash\hspace{0pt}}m{#1}}


\title{\vspace{-3.5cm} Fault Tolerance in Computational Systems - Report \\
\hrulefill\\
 Influence of outliers in a railway remote monitoring system%\\
%Giuseppe Lipari, Giorgio Buttazzo, Luca Abeni.
}
\author{Student: Vítor A. Morais\\
Supervisor: António Pina Martins }



\makenomenclature
\nomenclature{\textbf{AMI}}{Advanced Metering Infrastructure }
\nomenclature{\textbf{EMS}}{Energy Management System}
\nomenclature{\textbf{EIS}}{Energy Information Systems}
\nomenclature{\textbf{SG}}{Smart Grid}
\nomenclature{\textbf{IT}}{Information Technology}
\nomenclature{\textbf{AMR}}{Automatic Meter Readers}
\nomenclature{\textbf{SM}}{Smart Meter}
\nomenclature{\textbf{IEC}}{International Electrotechnical Commission}
\nomenclature{\textbf{EC}}{European Commission}
\nomenclature{\textbf{EU}}{European Union}
\nomenclature{\textbf{EGs}}{Expert Groups}
\nomenclature{\textbf{TSOs}}{Transmission System Operators}
\nomenclature{\textbf{DSOs}}{Distribution System Operators}
\nomenclature{\textbf{DNOs}}{Distribution Network Operators}

%2.2 to 2.6

\nomenclature{\textbf{WAN}}{Wide Area Network}
\nomenclature{\textbf{NAN}}{Neighborhood Area Network}
\nomenclature{\textbf{LAN}}{Local Area Network}
\nomenclature{\textbf{HAN}}{Home Area Network}
\nomenclature{\textbf{MAN}}{Metropolitan Area Network}
\nomenclature{\textbf{PEVs}}{Plug-in Electric Vehicles}
\nomenclature{\textbf{IHD}}{In-Home Displays}
\nomenclature{\textbf{MDMS}}{Meter Data Management System}
\nomenclature{\textbf{DR}}{Demand-Response}
\nomenclature{\textbf{SEIS}}{Smart Energy Management Systems}
\nomenclature{\textbf{ICT}}{Information and Communication Technologies}
\nomenclature{\textbf{O\&M}}{Operation and Management}
\nomenclature{\textbf{S2R}}{Shift2Rail program}
\nomenclature{\textbf{IP3}}{Innovation Programme 3 (of Shift2Rail)}
\nomenclature{\textbf{ODM}}{Operational Data Management}
\nomenclature{\textbf{UA}}{User Applications}
\nomenclature{\textbf{RDERMS}}{Railway dedicated Distributed Energy Resource Management System}

%chapter 3

\nomenclature{\textbf{SCADA}}{Supervisory Control and Data Acquisition}
\nomenclature{\textbf{PLC}}{Power Line Communication}
\nomenclature{\textbf{DCM}}{Data Collection Mechanism}
\nomenclature{\textbf{ADSL}}{Asymmetric  Digital  Subscriber  Line}
\nomenclature{\textbf{GSM}}{Global Systems Network}

\nomenclature{\textbf{SMS}}{Short Message Service}
\nomenclature{\textbf{CDMA}}{Code Division Multiple Access}
\nomenclature{\textbf{D-AMPS}}{Digital Advanced Mobile Phone Service}
\nomenclature{\textbf{RF}}{Radio Frequency}
\nomenclature{\textbf{WLAN}}{Wireless Local Area Network}
\nomenclature{\textbf{GPRS}}{General Packet Radio Service}
\nomenclature{\textbf{WiMAX}}{Worldwide Interoperability for Microwave Access}
\nomenclature{\textbf{IEEE}}{Institute of Electrical and Electronics Engineers}

\nomenclature{\textbf{ISO}}{International Organization for Standardization}
\nomenclature{\textbf{MAC}}{Media Access Control}
\nomenclature{\textbf{PHY}}{Physical Layer}
\nomenclature{\textbf{RFID}}{Radio Frequency Identification Devices}
\nomenclature{\textbf{ISM}}{Industrial, Scientific and Medical}
\nomenclature{\textbf{DSSS}}{Direct Sequence Spread Spectrum}

\nomenclature{\textbf{DASH7}}{Developers Alliance For Standards Harmonization of ISO 18000-7}
\nomenclature{\textbf{OFDM}}{Orthogonal Frequency-Division Multiplexing}
\nomenclature{\textbf{GMSK}}{Gaussian Minimum-Shift Keying}
\nomenclature{\textbf{LTE}}{Long Term Evolution}
\nomenclature{\textbf{QoS}}{Quality of Service}
\nomenclature{\textbf{BACnet}}{Building Automation and Control NETworks}

\nomenclature{\textbf{VSCP}}{Very Simple Control Protocol}
\nomenclature{\textbf{VSCP}}{Doctor of Philosophy}

\nomenclature{\textbf{ }}{ }
\nomenclature{\textbf{ }}{ }
\nomenclature{\textbf{ }}{ }



\begin{document}

\maketitle



%\chapter*{Abstract}
%Abstract goes here
%%
%%   Section 
%%
\tableofcontents


\printnomenclature

\chapter*{Symbols}
\begin{flushleft}

\begin{tabular}{l p{0.8\linewidth}}


kbps     & Kilobit per second (often used kbit/s or kb/s) - bit rate\\
Mbps     & Megabit per second (often used Mbit/s or Mb/s) - bit rate\\
Gbps     & Gigabit per second (often used Gbit/s or Gb/s) - bit rate\\
dB       & Decibel - Gain/Attenuation\\
kHz      & Kilohertz - Frequency\\
MHz      & Megahertz - Frequency\\
GHz      & Gigahertz - Frequency\\
km       & Kilometer - Distance\\
min      & Minute - Time\\


\end{tabular}
\end{flushleft}

\chapter{Introduction}
%\lipsum[4-4]
This chapter presents the context, motivation and document structure of the activity report of the work "monitoring photovoltaic power converters using Wireless Sensor Networks". 

\section{Context and motivation of PhD}

The railway system is responsible for 1.3\% of entire European energy consumption, \cite{iea-uic2016}. 
The discussion of the energy efficiency in railways is a grown topic due to its contribution to the global energy consumption.

The energy efficiency analysis and management requires a detailed mapping of the energy consumption/generation in the railway system. 

This detailed mapping of the energy flows should include, not only the rolling stock level but also the traction substations and the auxiliary services.

The knowledge of all the load curves permits the load prevision, peak shaving and energy cost optimization for all global railway system.

\section{Context and motivation of monitoring PV converters using WSN's}


This activity report is inserted in the scope of a microgeneration project of the SYSTEC Research Unit.
%Figure 1 in attachment presents the main architecture of this project. 
Currently, several work has been done in this project to implement a monitoring subsystem. This work focuses on data collection from each power converter and also on its control by sending references and actions. At the moment of this proposal, no work has been done to implement the monitoring feature in the PV converters. This way, a wireless network implementation is proposed to monitor the PV power converters.

The PV converter uses a non-isolated high gain DC/DC topology and has a PIC32 microcontroller which implements a MPPT algorithm in the control loop. It is currently possible to connect a device with wireless capabilities to the PV power converter. 

The converter is able to provide power and data. The data collected from the monitoring device is sent to an aggregator node for post processing and data storage. Currently, several work has already been done to have a reliable wired data aggregator/concentrator.
% (figure 2).
This subsystem is also responsible for sending user commands to the power converters and presenting the user interface in a webpage remotely accessed via a web browser. 




\section{Document structure}

This document is divided in 5 chapters, each of them incorporate the relevant subsections to present the subjects mentioned. 
%contains several subsections according to the subjects mentioned.

\begin{table}[!h]
    \label{tb:struct}
    \centering
    \caption{Document structure}
    \vspace{0.2em}
    \begin{tabular}{c|l}%{C{2cm}|C{9cm}}
    \textbf{Chapter} & \textbf{Title}                    \\ \hline
    1       &                   Introduction             \\ \hline
  %  2       &                   Railways Remote Monitoring Systems       \\ \hline
    2       &                   System Specification    \\ \hline
    3       &                   Development of Wireless Monitoring System    \\ \hline
    4       &                   Results and Discussion    \\ \hline
    5       &                   Conclusions and Future Work      \\ 
    \end{tabular}
\end{table}


\chapter{Railways Remote Monitoring Systems}
%\lipsum[4-4]
In this chapter it is an overview of the railway system where the outliers detection is expected to be studied.


%%%%%%%%%%%%%%%%%%%%%
%%   Smart metering Systems
%%%%%%%%%%%%%%%%%%%%%
\section{Smart Meters}



\section{Synthesis}

\chapter{Outliers Detection}
%\lipsum[4-4]
In this chapter it is  made the study of the state of the art of outliers and it's relevance in railways.


%%%%%%%%%%%%%%%%%%%%%
%%   Smart metering Systems
%%%%%%%%%%%%%%%%%%%%%
\section{Outliers detection definition}

\lipsum[1]


\section{Synthesis}

\chapter{Future Research}
%\lipsum[4-4]
In this chapter there are presented the future steps in research on outliers detection on railways WSN-based smart grid.


%%%%%%%%%%%%%%%%%%%%%
%%   Smart metering Systems
%%%%%%%%%%%%%%%%%%%%%
\section{Outliers detection definition}

\lipsum[1]


\section{Synthesis}

\chapter{Conclusion}
%\lipsum[1]

With this work, we propose and implement a solution for the wireless monitoring of power converters. The main challenge of this wireless monitoring is to obtain energy data of a system in a harsh environment. The motivation for this work is to study the implementation of a wireless monitoring system in railway environment, based on the outcomes of the implementation of such monitoring system in a renewable energy generation system.

The system specification was defined, with the construction of a System Requirement Specification (SRS) document as the main outcome. In addition, it was performed an overview on existing wireless communication systems, a market survey on available technologies, and the protocols, standards and communication KPI's was raised.

A specific technology was selected, on the family of wireless MCU's. The implementation stage complies with the definition of the hardware and software architecture. The communication results validates the acquisition of electric measurements from multiple PV power converters. 

The lack of data results is clear. Any of the communication KPI's was not evaluated due to the need of further development. This further development depends on a new solution for the interface between the power converter and the wireless node, since the communication link is highly affected by the noise of power converter.

For future work is clear the need of evaluating all the proposed communication KPI's. This task requires a new electronic board to interface the remote node. Complementary, the data concentrator should be improved to implement a local database and a serial request-response protocol to exchange data with the microgeneration system master.

We conclude that the methodology followed and the proposed solution validates the objective of this work, which is the wireless monitoring of a power converter.

\bibliographystyle{IEEEtran}
\bibliography{IEEEabrv,references}

%\bibliographystyle{plainnat}
%\bibliography{references}


\end{document}