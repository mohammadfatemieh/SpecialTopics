
\usepackage[english]{babel}


%\bibliographystyle{IEEEtran}

%\usepackage[square]{natbib}
\usepackage{graphicx}
\usepackage{framed}
\usepackage{multirow}
\usepackage{lipsum}  
\usepackage{verbatim}
\usepackage{amsmath}
\usepackage{longtable}
\usepackage{caption}
\usepackage{subcaption}


\usepackage{array}
\usepackage{adjustbox}


%\usepackage[oneside,width=17.5cm,height=24cm,left=2cm]{geometry}
%\usepackage[nolist,nohyperlinks]{acronym}

\usepackage[]{nomencl}
\usepackage[final]{pdfpages}

%\usepackage{biblatex}
%\addbibresource{references.bib}



\newcolumntype{L}[1]{>{\raggedright\let\newline\\\arraybackslash\hspace{0pt}}m{#1}}
\newcolumntype{C}[1]{>{\centering\let\newline\\\arraybackslash\hspace{0pt}}m{#1}}
\newcolumntype{R}[1]{>{\raggedleft\let\newline\\\arraybackslash\hspace{0pt}}m{#1}}

%%%  para ter capítulos xpto
%%%  https://hstuart.dk/2007/05/21/styling-the-chapter/


\usepackage{tikz, blindtext}
\usepackage{kpfonts}
\usepackage[explicit]{titlesec}
\usepackage{xcolor}
\definecolor{FEUP_color}{RGB}{140,45,25}

\newcommand*\chapterlabel{}
\titleformat{\chapter}
{\gdef\chapterlabel{}
	\normalfont\sffamily\huge\bfseries\scshape}
{\gdef\chapterlabel{\thechapter\ }}{0pt}
{\begin{tikzpicture}[remember picture,overlay]
	\node[yshift=-6cm] at (current page.north west)
	{\begin{tikzpicture}[remember picture, overlay]
		%\draw[fill=white] (-1,0) rectangle
	%	(1.2\paperwidth,8cm);
		\node[anchor=west,xshift=3cm,rectangle,
		rounded corners=1pt,inner sep=11pt,
		fill=black]
		{\color{white}\chapterlabel#1};
		\end{tikzpicture}
	};
\end{tikzpicture}
}
\titlespacing*{\chapter}{0pt}{50pt}{50pt}

\makenomenclature
\nomenclature{\textbf{AMI}}{Advanced Metering Infrastructure }
\nomenclature{\textbf{EMS}}{Energy Management System}
\nomenclature{\textbf{EIS}}{Energy Information Systems}
\nomenclature{\textbf{SG}}{Smart Grid}
\nomenclature{\textbf{IT}}{Information Technology}
\nomenclature{\textbf{AMR}}{Automatic Meter Readers}
\nomenclature{\textbf{SM}}{Smart Meter}
\nomenclature{\textbf{IEC}}{International Electrotechnical Commission}
\nomenclature{\textbf{EC}}{European Commission}
\nomenclature{\textbf{EU}}{European Union}
\nomenclature{\textbf{EGs}}{Expert Groups}
\nomenclature{\textbf{TSOs}}{Transmission System Operators}
\nomenclature{\textbf{DSOs}}{Distribution System Operators}
\nomenclature{\textbf{DNOs}}{Distribution Network Operators}

%2.2 to 2.6

\nomenclature{\textbf{WAN}}{Wide Area Network}
\nomenclature{\textbf{NAN}}{Neighborhood Area Network}
\nomenclature{\textbf{LAN}}{Local Area Network}
\nomenclature{\textbf{HAN}}{Home Area Network}
\nomenclature{\textbf{MAN}}{Metropolitan Area Network}
\nomenclature{\textbf{PEVs}}{Plug-in Electric Vehicles}
\nomenclature{\textbf{IHD}}{In-Home Displays}
\nomenclature{\textbf{MDMS}}{Meter Data Management System}
\nomenclature{\textbf{DR}}{Demand-Response}
\nomenclature{\textbf{SEIS}}{Smart Energy Management Systems}
\nomenclature{\textbf{ICT}}{Information and Communication Technologies}
\nomenclature{\textbf{O\&M}}{Operation and Management}
\nomenclature{\textbf{S2R}}{Shift2Rail program}
\nomenclature{\textbf{IP3}}{Innovation Programme 3 (of Shift2Rail)}
\nomenclature{\textbf{ODM}}{Operational Data Management}
\nomenclature{\textbf{UA}}{User Applications}
\nomenclature{\textbf{RDERMS}}{Railway dedicated Distributed Energy Resource Management System}

%chapter 3

\nomenclature{\textbf{SCADA}}{Supervisory Control and Data Acquisition}
\nomenclature{\textbf{PLC}}{Power Line Communication}
\nomenclature{\textbf{DCM}}{Data Collection Mechanism}
\nomenclature{\textbf{ADSL}}{Asymmetric  Digital  Subscriber  Line}
\nomenclature{\textbf{GSM}}{Global Systems Network}

\nomenclature{\textbf{SMS}}{Short Message Service}
\nomenclature{\textbf{CDMA}}{Code Division Multiple Access}
\nomenclature{\textbf{D-AMPS}}{Digital Advanced Mobile Phone Service}
\nomenclature{\textbf{RF}}{Radio Frequency}
\nomenclature{\textbf{WLAN}}{Wireless Local Area Network}
\nomenclature{\textbf{GPRS}}{General Packet Radio Service}
\nomenclature{\textbf{WiMAX}}{Worldwide Interoperability for Microwave Access}
\nomenclature{\textbf{IEEE}}{Institute of Electrical and Electronics Engineers}

\nomenclature{\textbf{ISO}}{International Organization for Standardization}
\nomenclature{\textbf{MAC}}{Media Access Control}
\nomenclature{\textbf{PHY}}{Physical Layer}
\nomenclature{\textbf{RFID}}{Radio Frequency Identification Devices}
\nomenclature{\textbf{ISM}}{Industrial, Scientific and Medical}
\nomenclature{\textbf{DSSS}}{Direct Sequence Spread Spectrum}

\nomenclature{\textbf{DASH7}}{Developers Alliance For Standards Harmonization of ISO 18000-7}
\nomenclature{\textbf{OFDM}}{Orthogonal Frequency-Division Multiplexing}
\nomenclature{\textbf{GMSK}}{Gaussian Minimum-Shift Keying}
\nomenclature{\textbf{LTE}}{Long Term Evolution}
\nomenclature{\textbf{QoS}}{Quality of Service}
\nomenclature{\textbf{BACnet}}{Building Automation and Control NETworks}

\nomenclature{\textbf{VSCP}}{Very Simple Control Protocol}
\nomenclature{\textbf{VSCP}}{Doctor of Philosophy}

\nomenclature{\textbf{ }}{ }
\nomenclature{\textbf{ }}{ }
\nomenclature{\textbf{ }}{ }




